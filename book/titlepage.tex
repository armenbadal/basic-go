\begin{titlepage}
\titlefont

\begin{center}
{\Large Ա}{\large ՐՄԵՆ}~~{\Large Բ}{\large ԱԴԱԼՅԱՆ}
\end{center}

\vskip4cm

\begin{center}
\begin{tabular}{c}
{\fontsize{52}{52}\selectfont\textbf{ԻՆՏԵՐՊՐԵՏԱՏՈՐ}} \\[6pt]
{\Large ալգորիթմական լեզվի իրականացումը՝} \\[4pt]
{\Large մի մեծ վարժություն}
\end{tabular}
\end{center}

\vfill

\begin{center}
ԵՐԵՒԱՆ

\MakeUppercase{\romannumeral\year{}} 
\end{center}

\end{titlepage}



\begin{titlepage}

\begingroup
\small\sffamily\linespread{1}
\begin{tabular}{rl}
ԳՄԴ & xxx.xx:xxx.xx \\
ՀՏԴ & xx.xx \\
  Ա & xx \\
\end{tabular}
\endgroup

\vskip2cm

\begin{tabular}{rl}
     & Բադալյան Արմեն \\
Ա xx & \textsl{Ինտերպրետատոր. ալգորիթմական լեզվի իրականացումը՝ մի մեծ} \\
     & \textsl{վարժություն} / Արմեն Բադալյան, Երեւան, 2026, xxx էջ:
\end{tabular}

\bigskip

\begingroup\small
Այս գիրքը ծրագրավորման լեզվի ինտերպրետատոր ստեղծելու մասին է։ Հեղինակը պարզ
լեզվով ու բազմաթիվ օրինակներով պատմում է բալ լեզվի ինտերպրետատորի իրականացման
բոլոր քայլերը։ Գիրքը նախատեսված է ինֆորմատիկայի ֆակուլտետի ուսանողների համար,
սակայն կարող է օգտակար լինել նաև բոլոր նրանց, ովքեր հետաքրքրվում են ծրագրավորման 
լեզուների նախագծմամբ ու իրականացմամբ։
\endgroup

\vfill
\textsf{\copyright\ Արմեն Բադալյան, \the\year{}։}

\end{titlepage}

